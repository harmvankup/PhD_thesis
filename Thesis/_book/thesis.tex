% This is the Reed College LaTeX thesis template. Most of the work
% for the document class was done by Sam Noble (SN), as well as this
% template. Later comments etc. by Ben Salzberg (BTS). Additional
% restructuring and APA support by Jess Youngberg (JY).
% Your comments and suggestions are more than welcome; please email
% them to cus@reed.edu
%
% See https://www.reed.edu/cis/help/LaTeX/index.html for help. There are a
% great bunch of help pages there, with notes on
% getting started, bibtex, etc. Go there and read it if you're not
% already familiar with LaTeX.
%
% Any line that starts with a percent symbol is a comment.
% They won't show up in the document, and are useful for notes
% to yourself and explaining commands.
% Commenting also removes a line from the document;
% very handy for troubleshooting problems. -BTS

% As far as I know, this follows the requirements laid out in
% the 2002-2003 Senior Handbook. Ask a librarian to check the
% document before binding. -SN

%%
%% Preamble
%%
% \documentclass{<something>} must begin each LaTeX document
\documentclass[12pt,twoside]{reedthesis}
% Packages are extensions to the basic LaTeX functions. Whatever you
% want to typeset, there is probably a package out there for it.
% Chemistry (chemtex), screenplays, you name it.
% Check out CTAN to see: https://www.ctan.org/
%%
\usepackage{graphicx,latexsym}
\usepackage{amsmath}
\usepackage{amssymb,amsthm}
\usepackage{longtable,booktabs,setspace}
\usepackage{chemarr} %% Useful for one reaction arrow, useless if you're not a chem major
\usepackage[hyphens]{url}
% Added by CII
\usepackage{hyperref}
\usepackage{lmodern}
\usepackage{float}
\floatplacement{figure}{H}
% Thanks, @Xyv
\usepackage{calc}
% End of CII addition
\usepackage{rotating}

% Next line commented out by CII
\usepackage{natbib}
% Comment out the natbib line above and uncomment the following two lines to use the new
% biblatex-chicago style, for Chicago A. Also make some changes at the end where the
% bibliography is included.
%\usepackage{biblatex-chicago}
\bibliography{thesis}


% Added by CII (Thanks, Hadley!)
% Use ref for internal links
\renewcommand{\hyperref}[2][???]{\autoref{#1}}

\def\chapterautorefname{Chapter}
\def\sectionautorefname{Section}
\def\subsectionautorefname{Subsection}
% End of CII addition

% Added by CII
\usepackage{caption}
\captionsetup{width=5in}
% End of CII addition

% \usepackage{times} % other fonts are available like times, bookman, charter, palatino

% Syntax highlighting #22
  \usepackage{color}
  \usepackage{fancyvrb}
  \newcommand{\VerbBar}{|}
  \newcommand{\VERB}{\Verb[commandchars=\\\{\}]}
  \DefineVerbatimEnvironment{Highlighting}{Verbatim}{commandchars=\\\{\}}
  % Add ',fontsize=\small' for more characters per line
  \usepackage{framed}
  \definecolor{shadecolor}{RGB}{248,248,248}
  \newenvironment{Shaded}{\begin{snugshade}}{\end{snugshade}}
  \newcommand{\AlertTok}[1]{\textcolor[rgb]{0.94,0.16,0.16}{#1}}
  \newcommand{\AnnotationTok}[1]{\textcolor[rgb]{0.56,0.35,0.01}{\textbf{\textit{#1}}}}
  \newcommand{\AttributeTok}[1]{\textcolor[rgb]{0.13,0.29,0.53}{#1}}
  \newcommand{\BaseNTok}[1]{\textcolor[rgb]{0.00,0.00,0.81}{#1}}
  \newcommand{\BuiltInTok}[1]{#1}
  \newcommand{\CharTok}[1]{\textcolor[rgb]{0.31,0.60,0.02}{#1}}
  \newcommand{\CommentTok}[1]{\textcolor[rgb]{0.56,0.35,0.01}{\textit{#1}}}
  \newcommand{\CommentVarTok}[1]{\textcolor[rgb]{0.56,0.35,0.01}{\textbf{\textit{#1}}}}
  \newcommand{\ConstantTok}[1]{\textcolor[rgb]{0.56,0.35,0.01}{#1}}
  \newcommand{\ControlFlowTok}[1]{\textcolor[rgb]{0.13,0.29,0.53}{\textbf{#1}}}
  \newcommand{\DataTypeTok}[1]{\textcolor[rgb]{0.13,0.29,0.53}{#1}}
  \newcommand{\DecValTok}[1]{\textcolor[rgb]{0.00,0.00,0.81}{#1}}
  \newcommand{\DocumentationTok}[1]{\textcolor[rgb]{0.56,0.35,0.01}{\textbf{\textit{#1}}}}
  \newcommand{\ErrorTok}[1]{\textcolor[rgb]{0.64,0.00,0.00}{\textbf{#1}}}
  \newcommand{\ExtensionTok}[1]{#1}
  \newcommand{\FloatTok}[1]{\textcolor[rgb]{0.00,0.00,0.81}{#1}}
  \newcommand{\FunctionTok}[1]{\textcolor[rgb]{0.13,0.29,0.53}{\textbf{#1}}}
  \newcommand{\ImportTok}[1]{#1}
  \newcommand{\InformationTok}[1]{\textcolor[rgb]{0.56,0.35,0.01}{\textbf{\textit{#1}}}}
  \newcommand{\KeywordTok}[1]{\textcolor[rgb]{0.13,0.29,0.53}{\textbf{#1}}}
  \newcommand{\NormalTok}[1]{#1}
  \newcommand{\OperatorTok}[1]{\textcolor[rgb]{0.81,0.36,0.00}{\textbf{#1}}}
  \newcommand{\OtherTok}[1]{\textcolor[rgb]{0.56,0.35,0.01}{#1}}
  \newcommand{\PreprocessorTok}[1]{\textcolor[rgb]{0.56,0.35,0.01}{\textit{#1}}}
  \newcommand{\RegionMarkerTok}[1]{#1}
  \newcommand{\SpecialCharTok}[1]{\textcolor[rgb]{0.81,0.36,0.00}{\textbf{#1}}}
  \newcommand{\SpecialStringTok}[1]{\textcolor[rgb]{0.31,0.60,0.02}{#1}}
  \newcommand{\StringTok}[1]{\textcolor[rgb]{0.31,0.60,0.02}{#1}}
  \newcommand{\VariableTok}[1]{\textcolor[rgb]{0.00,0.00,0.00}{#1}}
  \newcommand{\VerbatimStringTok}[1]{\textcolor[rgb]{0.31,0.60,0.02}{#1}}
  \newcommand{\WarningTok}[1]{\textcolor[rgb]{0.56,0.35,0.01}{\textbf{\textit{#1}}}}

% To pass between YAML and LaTeX the dollar signs are added by CII
\title{Untitled}
\author{Harm Noël van Kuppevelt}
% The month and year that you submit your FINAL draft TO THE LIBRARY (May or December)
\date{May 2025}
\division{Von der Fakultät für Umwelt und Naturwissenschaften}
\advisor{M Hupfer}
\institution{der Brandenburgischen Technischen Universität Cottbus-Senftenberg}
\degree{Doktors der Naturwissenschaften (Dr.~rer. nat.)}
%If you have two advisors for some reason, you can use the following
% Uncommented out by CII
\quote{\emph{Planet Earth is blue}\\
\emph{and there's nothing we can do}\\
\textbf{David Bowie}}

% End of CII addition

%%% Remember to use the correct department!
\department{Leibniz-Institut für Gewässerökologie und Binnenfischerei}
% if you're writing a thesis in an interdisciplinary major,
% uncomment the line below and change the text as appropriate.
% check the Senior Handbook if unsure.
%\thedivisionof{The Established Interdisciplinary Committee for}
% if you want the approval page to say "Approved for the Committee",
% uncomment the next line
%\approvedforthe{Committee}

% Added by CII
%%% Copied from knitr
%% maxwidth is the original width if it's less than linewidth
%% otherwise use linewidth (to make sure the graphics do not exceed the margin)
\makeatletter
\def\maxwidth{ %
  \ifdim\Gin@nat@width>\linewidth
    \linewidth
  \else
    \Gin@nat@width
  \fi
}
\makeatother

% From {rticles}
\newlength{\csllabelwidth}
\setlength{\csllabelwidth}{3em}
\newlength{\cslhangindent}
\setlength{\cslhangindent}{1.5em}
% for Pandoc 2.8 to 2.10.1
\newenvironment{cslreferences}%
  {}%
  {\par}
% For Pandoc 2.11+
% As noted by @mirh [2] is needed instead of [3] for 2.12
\newenvironment{CSLReferences}[2] % #1 hanging-ident, #2 entry spacing
 {% don't indent paragraphs
  \setlength{\parindent}{0pt}
  % turn on hanging indent if param 1 is 1
  \ifodd #1 \everypar{\setlength{\hangindent}{\cslhangindent}}\ignorespaces\fi
  % set entry spacing
  \ifnum #2 > 0
  \setlength{\parskip}{#2\baselineskip}
  \fi
 }%
 {}
\usepackage{calc} % for calculating minipage widths
\newcommand{\CSLBlock}[1]{#1\hfill\break}
\newcommand{\CSLLeftMargin}[1]{\parbox[t]{\csllabelwidth}{#1}}
\newcommand{\CSLRightInline}[1]{\parbox[t]{\linewidth - \csllabelwidth}{#1}}
\newcommand{\CSLIndent}[1]{\hspace{\cslhangindent}#1}

\renewcommand{\contentsname}{Table of Contents}
% End of CII addition

\setlength{\parskip}{0pt}

% Added by CII

\providecommand{\tightlist}{%
  \setlength{\itemsep}{0pt}\setlength{\parskip}{0pt}}

\Acknowledgements{

}

\Zusammenfassung{
This is the abstract in German

\par

Second paragraph of abstract starts here.
}

\Preface{

}

\Summary{
This is the document

\par

Second paragraph of abstract starts here.
}

	\usepackage{setspace}
	\onehalfspacing
% End of CII addition
%%
%% End Preamble
%%
%

\begin{document}

% Everything below added by CII
  \maketitle

\frontmatter % this stuff will be roman-numbered
\pagestyle{empty} % this removes page numbers from the frontmatter






  \hypersetup{linkcolor=black}
  \setcounter{secnumdepth}{2}
  \setcounter{tocdepth}{2}
  \tableofcontents



  \begin{summary}
    This is the document

    \par

    Second paragraph of abstract starts here.
  \end{summary}

  \begin{zusammenfassung}
    This is the abstract in German

    \par

    Second paragraph of abstract starts here.
  \end{zusammenfassung}

\mainmatter % here the regular arabic numbering starts
\pagestyle{fancyplain} % turns page numbering back on

\chapter{Introduction}\label{Intro}

\section{The Phosphorus problem}\label{the-phosphorus-problem}

A great introduction explaining the global challenge and myriad of threats to our health and that of a sustainable way of living.
Introduce the Geological cylces of Life, from small to big, Carbon,Water, Nitrogen and of course phosphorus.

The evolution of humans on earth was followed by unprecedented changes in the Earths environment and disruption of the vital biogeochemical cycles underpinning the regenerative nature that characterized the climatologically stable Holocene epoch. As a species that can recognize the importance of live and it's essential building blocks, we bear the responsibility of a fair and sustainable distribution of the earths resources. Unfortunately, externalizing impacts in favor of short-term profit has become {[}@richardson2023{]}

\begin{figure}

{\centering \includegraphics[width=0.5\linewidth]{figure/pBoundries} 

}

\caption{P boundries}\label{fig:boundries}
\end{figure}

\section{Phosphorus cycling in freshwater sediments}\label{phosphorus-cycling-in-freshwater-sediments}

Talk about all the different forms P enters the sediment, and the early diagenesis.

The retention of P in sediments occurs through its binding to the solid phase via biological and chemical precipitation {[}@boers1998; @oconnell2020; @parsons2017{]}. Burial P pools accumulate in sediment following years of nutrient enrichment. The speciation of sediment P is highly dependent on lake conditions and undergoes significant changes during early diagenesis, where remobilisation of labile forms of P occurs, whereas only the stable forms of solid-bound P are buried long-term (Boers et al., 1998; Emerson, 1976). Initially, the sedimentation of P incorporated into organic matter (OM) by primary producers is an important influx of P to lake sediments, especially in eutrophic lakes. However, the long-term burial of OM-bound P is constrained by remineralisation processes in the sediment (Boers et al., 1998). Secondly, P can adsorb to iron (Fe) hydroxides (Gunnars et al., 2002) and bind to OM to form organic Fe-P complexes (Fe(III)-OM-P) (Schwertmann and Murad, 1988). The precipitation of these ferric iron-bound P forms (Fe(III)-P) can be a major internal sink of P in lakes with naturally high Fe content (Hupfer and Lewandowski, 2008; Reitzel et al., 2005) or those artificially treated with Fe (Kleeberg et al., 2012; Münch et al., 2024). Under the reducing conditions induced by organic matter decomposition in the sediment, Fe(III) is reduced to Fe(II), leading to the release of bound P and preventing the long-term burial of Fe(III)-bound P. However, P can be sequestered long-term in the form of the Fe(II) mineral vivianite (Fe(II)3(PO4)2·8H2O) (Rothe et al., 2016), which has been identified as a major form of burial P in eutrophic, high-Fe, and nonsulphidic freshwater systems (Dijkstra et al., 2018; Kubeneck et al., 2021; O'Connell et al., 2015; Rothe, 2016).

\section{Sulfur biogeochemistry}\label{sulfur-biogeochemistry}

(Due to eutrophication, i.e.~a primarily enhanced P supply, the P-binding capacity of a sediment will be exceeded leading to a higher P mobility and less or no vivianite formation. A higher productivity leads to a higher OM supply toward the sediment which has consequences for the formation of vivianite. First, there is a higher demand for oxidants leading to a deterioration of redox conditions and higher reduction rates of ferric Fe and SO24 (Holmer \& Storkholm, 2001). Second, there is more S2-- produced because OM is specifically enriched in S compared to Fe (Redfield ratio: C106N16P1S0.7Fe0.05, (Stumm \& Morgan, 1981)). Sulphides are formed by both desulphuration and dissimilatory sulphate reduction leading to a higher degree of sediment sulphidization. 6

3.6 Conclusions The former can be quite significant in overall sedimentary hydrogen sulphide production, e.g.~5.1 - 53 \% (Dunnette et al., 1985). Moreover, eutrophication is often accompanied by considerable inputs of SO24 leading to its higher availability and high rates of its consumption (Holmer \& Storkholm, 2001; Zak et al., 2006). Third, the OM itself can react with Fe forming a metal organic complex (Lalonde et al., 2012). The higher the sedimentary S:Fe ratio, the less reactive Fe seems to be available reducing the potential of vivianite to form (Fig. 3.5) because more Fe is bound in sulphidic form. Thus, under eutrophic conditions there is a negative feedback evolving through the enhanced supply of OM lowering the sedimentary P retention capacity due to less vivianite. Aquatic systems naturally high in reactive Fe may compensate better for a eutrophication induced decrease in P retention than systems low in Fe. This implies, that an artificial supply of Fe to systems with a high level in OM, P and SO24 can be used as a successful measure of lake restoration leading to increased P retention through vivianite formation (Kleeberg et al., 2013; Rothe et al., 2014). To ensure a lasting effect on P burial, Fe has to be supplied in surplus compensating for the losses through FeSx formation (Kleeberg et al., 2013) and the reaction with OM (Lalonde et al., 2012). At which magnitude vivianite finally forms in different types of sediments depends on multiple factors and remains to be further investigated. The formation of the mineral is also controlled by the availability of OM rich in P, the concomittant liberation of Fe2+ and PO34 into the pore voids of the sediment, the activity of microorganisms and resorption of PO34 onto the surface of remaining iron(oxyhydr)oxides.) Rothe 2015

\section{Vivianite stability}\label{vivianite-stability}

Start at the beginning, vivianite characteristics.
Next, the formation requirments and kinetics {[}@paskin2024{]}
Then, the stability under oxygen conditions
And then the part about the sulfidation

\begin{equation}
  \mathrm{Fe_3(PO_{4})_2\bullet H_2O(s) + 2H^+} \rightleftharpoons \mathrm{3Fe^{2+} + 2HPO_4^{2-} + 8H_2O}
  \label{eq:vivianitedissolve}
\end{equation}

\begin{equation}
  \mathrm{Fe^{2+} + xHS^-} \rightleftharpoons \mathrm{FeS_x(s) + xH^+}
  \label{eq:sulfideform}
\end{equation}

\section{Research objectives and outline}\label{research-objectives-and-outline}

\chapter{R Markdown Basics}\label{rmd-basics}

Here is a brief introduction into using \emph{R Markdown}. \emph{Markdown} is a simple formatting syntax for authoring HTML, PDF, and MS Word documents. \emph{R Markdown} provides the flexibility of \emph{Markdown} with the implementation of \textbf{R} input and output. For more details on using \emph{R Markdown} see \url{https://rmarkdown.rstudio.com}.

Be careful with your spacing in \emph{Markdown} documents. While whitespace largely is ignored, it does at times give \emph{Markdown} signals as to how to proceed. As a habit, try to keep everything left aligned whenever possible, especially as you type a new paragraph. In other words, there is no need to indent basic text in the Rmd document (in fact, it might cause your text to do funny things if you do).

\section{Lists}\label{lists}

It's easy to create a list. It can be unordered like

\begin{itemize}
\tightlist
\item
  Item 1
\item
  Item 2
\end{itemize}

or it can be ordered like

\begin{enumerate}
\def\labelenumi{\arabic{enumi}.}
\tightlist
\item
  Item 1
\item
  Item 2
\end{enumerate}

Notice that I intentionally mislabeled Item 2 as number 4. \emph{Markdown} automatically figures this out! You can put any numbers in the list and it will create the list. Check it out below.

To create a sublist, just indent the values a bit (at least four spaces or a tab). (Here's one case where indentation is key!)

\begin{enumerate}
\def\labelenumi{\arabic{enumi}.}
\tightlist
\item
  Item 1
\item
  Item 2
\item
  Item 3

  \begin{itemize}
  \tightlist
  \item
    Item 3a
  \item
    Item 3b
  \end{itemize}
\end{enumerate}

\section{Line breaks}\label{line-breaks}

Make sure to add white space between lines if you'd like to start a new paragraph. Look at what happens below in the outputted document if you don't:

Here is the first sentence. Here is another sentence. Here is the last sentence to end the paragraph.
This should be a new paragraph.

\emph{Now for the correct way:}

Here is the first sentence. Here is another sentence. Here is the last sentence to end the paragraph.

This should be a new paragraph.

\section{R chunks}\label{r-chunks}

When you click the \textbf{Knit} button above a document will be generated that includes both content as well as the output of any embedded \textbf{R} code chunks within the document. You can embed an \textbf{R} code chunk like this (\texttt{cars} is a built-in \textbf{R} dataset):

\section{Inline code}\label{inline-code}

If you'd like to put the results of your analysis directly into your discussion, add inline code like this:

\begin{quote}
The \texttt{cos} of \(2 \pi\) is 1.
\end{quote}

Another example would be the direct calculation of the standard deviation:

\begin{quote}
The standard deviation of \texttt{speed} in \texttt{cars} is 5.2876444.
\end{quote}

One last neat feature is the use of the \texttt{ifelse} conditional statement which can be used to output text depending on the result of an \textbf{R} calculation:

\begin{quote}
The standard deviation is less than 6.
\end{quote}

Note the use of \texttt{\textgreater{}} here, which signifies a quotation environment that will be indented.

As you see with \texttt{\$2\ \textbackslash{}pi\$} above, mathematics can be added by surrounding the mathematical text with dollar signs. More examples of this are in \hyperref[math-sci]{Mathematics and Science} if you uncomment the code in \hyperref[math]{Math}.

\section{Including plots}\label{including-plots}

You can also embed plots. For example, here is a way to use the base \textbf{R} graphics package to produce a plot using the built-in \texttt{pressure} dataset:

\includegraphics{thesis_files/figure-latex/pressure-1.pdf}

Note that the \texttt{echo=FALSE} parameter was added to the code chunk to prevent printing of the \textbf{R} code that generated the plot. There are plenty of other ways to add chunk options (like \texttt{fig.height} and \texttt{fig.width} in the chunk above). More information is available at \url{https://yihui.org/knitr/options/}.

Another useful chunk option is the setting of \texttt{cache=TRUE} as you see here. If document rendering becomes time consuming due to long computations or plots that are expensive to generate you can use knitr caching to improve performance. Later in this file, you'll see a way to reference plots created in \textbf{R} or external figures.

\section{Loading and exploring data}\label{loading-and-exploring-data}

Included in this template is a file called \texttt{flights.csv}. This file includes a subset of the larger dataset of information about all flights that departed from Seattle and Portland in 2014. More information about this dataset and its \textbf{R} package is available at \url{https://github.com/ismayc/pnwflights14}. This subset includes only Portland flights and only rows that were complete with no missing values. Merges were also done with the \texttt{airports} and \texttt{airlines} data sets in the \texttt{pnwflights14} package to get more descriptive airport and airline names.

We can load in this data set using the following commands:

\begin{Shaded}
\begin{Highlighting}[]
\CommentTok{\# flights.csv is in the data directory}
\NormalTok{flights\_path }\OtherTok{\textless{}{-}}\NormalTok{ here}\SpecialCharTok{::}\FunctionTok{here}\NormalTok{(}\StringTok{"Thesis"}\NormalTok{,}\StringTok{"data"}\NormalTok{, }\StringTok{"flights.csv"}\NormalTok{)}
\CommentTok{\# string columns will be read in as strings and not factors now}
\NormalTok{flights }\OtherTok{\textless{}{-}} \FunctionTok{read.csv}\NormalTok{(flights\_path, }\AttributeTok{stringsAsFactors =} \ConstantTok{FALSE}\NormalTok{)}
\end{Highlighting}
\end{Shaded}

The data is now stored in the data frame called \texttt{flights} in \textbf{R}. To get a better feel for the variables included in this dataset we can use a variety of functions. Here we can see the dimensions (rows by columns) and also the names of the columns.

\begin{Shaded}
\begin{Highlighting}[]
\FunctionTok{dim}\NormalTok{(flights)}
\end{Highlighting}
\end{Shaded}

\begin{verbatim}
[1] 12649    16
\end{verbatim}

\begin{Shaded}
\begin{Highlighting}[]
\FunctionTok{names}\NormalTok{(flights)}
\end{Highlighting}
\end{Shaded}

\begin{verbatim}
 [1] "month"        "day"          "dep_time"     "dep_delay"    "arr_time"    
 [6] "arr_delay"    "carrier"      "tailnum"      "flight"       "dest"        
[11] "air_time"     "distance"     "hour"         "minute"       "carrier_name"
[16] "dest_name"   
\end{verbatim}

Another good idea is to take a look at the dataset in table form. With this dataset having more than 20,000 rows, we won't explicitly show the results of the command here. I recommend you enter the command into the Console \textbf{\emph{after}} you have run the \textbf{R} chunks above to load the data into \textbf{R}.

\begin{Shaded}
\begin{Highlighting}[]
\FunctionTok{View}\NormalTok{(flights)}
\end{Highlighting}
\end{Shaded}

While not required, it is highly recommended you use the \texttt{dplyr} package to manipulate and summarize your data set as needed. It uses a syntax that is easy to understand using chaining operations. Below I've created a few examples of using \texttt{dplyr} to get information about the Portland flights in 2014. You will also see the use of the \texttt{ggplot2} package, which produces beautiful, high-quality academic visuals.

We begin by checking to ensure that needed packages are installed and then we load them into our current working environment:

\begin{Shaded}
\begin{Highlighting}[]
\CommentTok{\# List of packages required for this analysis}
\NormalTok{pkg }\OtherTok{\textless{}{-}} \FunctionTok{c}\NormalTok{(}\StringTok{"dplyr"}\NormalTok{, }\StringTok{"ggplot2"}\NormalTok{, }\StringTok{"knitr"}\NormalTok{, }\StringTok{"bookdown"}\NormalTok{)}
\CommentTok{\# Check if packages are not installed and assign the}
\CommentTok{\# names of the packages not installed to the variable new.pkg}
\NormalTok{new.pkg }\OtherTok{\textless{}{-}}\NormalTok{ pkg[}\SpecialCharTok{!}\NormalTok{(pkg }\SpecialCharTok{\%in\%} \FunctionTok{installed.packages}\NormalTok{())]}
\CommentTok{\# If there are any packages in the list that aren\textquotesingle{}t installed,}
\CommentTok{\# install them}
\ControlFlowTok{if}\NormalTok{ (}\FunctionTok{length}\NormalTok{(new.pkg)) \{}
  \FunctionTok{install.packages}\NormalTok{(new.pkg, }\AttributeTok{repos =} \StringTok{"https://cran.rstudio.com"}\NormalTok{)}
\NormalTok{\}}
\CommentTok{\# Load packages}
\FunctionTok{library}\NormalTok{(thesisdown)}
\FunctionTok{library}\NormalTok{(dplyr)}
\FunctionTok{library}\NormalTok{(ggplot2)}
\FunctionTok{library}\NormalTok{(knitr)}
\end{Highlighting}
\end{Shaded}

\clearpage

The example we show here does the following:

\begin{itemize}
\item
  Selects only the \texttt{carrier\_name} and \texttt{arr\_delay} from the \texttt{flights} dataset and then assigns this subset to a new variable called \texttt{flights2}.
\item
  Using \texttt{flights2}, we determine the largest arrival delay for each of the carriers.
\end{itemize}

\begin{Shaded}
\begin{Highlighting}[]
\NormalTok{flights2 }\OtherTok{\textless{}{-}}\NormalTok{ flights }\SpecialCharTok{\%\textgreater{}\%}
  \FunctionTok{select}\NormalTok{(carrier\_name, arr\_delay)}
\NormalTok{max\_delays }\OtherTok{\textless{}{-}}\NormalTok{ flights2 }\SpecialCharTok{\%\textgreater{}\%}
  \FunctionTok{group\_by}\NormalTok{(carrier\_name) }\SpecialCharTok{\%\textgreater{}\%}
  \FunctionTok{summarize}\NormalTok{(}\AttributeTok{max\_arr\_delay =} \FunctionTok{max}\NormalTok{(arr\_delay, }\AttributeTok{na.rm =} \ConstantTok{TRUE}\NormalTok{))}
\end{Highlighting}
\end{Shaded}

A useful function in the \texttt{knitr} package for making nice tables in \emph{R Markdown} is called \texttt{kable}. It is much easier to use than manually entering values into a table by copying and pasting values into Excel or LaTeX. This again goes to show how nice reproducible documents can be! (Note the use of \texttt{results="asis"}, which will produce the table instead of the code to create the table.) The \texttt{caption.short} argument is used to include a shorter title to appear in the List of Tables.

\begin{Shaded}
\begin{Highlighting}[]
\FunctionTok{kable}\NormalTok{(max\_delays,}
  \AttributeTok{col.names =} \FunctionTok{c}\NormalTok{(}\StringTok{"Airline"}\NormalTok{, }\StringTok{"Max Arrival Delay"}\NormalTok{),}
  \AttributeTok{caption =} \StringTok{"Maximum Delays by Airline"}\NormalTok{,}
  \AttributeTok{caption.short =} \StringTok{"Max Delays by Airline"}\NormalTok{,}
  \AttributeTok{longtable =} \ConstantTok{TRUE}\NormalTok{,}
  \AttributeTok{booktabs =} \ConstantTok{TRUE}
\NormalTok{)}
\end{Highlighting}
\end{Shaded}

\begin{longtable}[t]{lr}
\caption[Max Delays by Airline]{\label{tab:maxdelays}Maximum Delays by Airline}\\
\toprule
Airline & Max Arrival Delay\\
\midrule
Alaska Airlines Inc. & 338\\
American Airlines Inc. & 1539\\
Delta Air Lines Inc. & 371\\
Frontier Airlines Inc. & 166\\
Hawaiian Airlines Inc. & 116\\
\addlinespace
JetBlue Airways & 256\\
SkyWest Airlines Inc. & 321\\
Southwest Airlines Co. & 315\\
US Airways Inc. & 347\\
United Air Lines Inc. & 319\\
\addlinespace
Virgin America & 366\\
\bottomrule
\end{longtable}

The last two options make the table a little easier-to-read.

We can further look into the properties of the largest value here for American Airlines Inc.~To do so, we can isolate the row corresponding to the arrival delay of 1539 minutes for American in our original \texttt{flights} dataset.

\begin{Shaded}
\begin{Highlighting}[]
\NormalTok{flights }\SpecialCharTok{\%\textgreater{}\%}
  \FunctionTok{filter}\NormalTok{(}
\NormalTok{    arr\_delay }\SpecialCharTok{==} \DecValTok{1539}\NormalTok{,}
\NormalTok{    carrier\_name }\SpecialCharTok{==} \StringTok{"American Airlines Inc."}
\NormalTok{  ) }\SpecialCharTok{\%\textgreater{}\%}
  \FunctionTok{select}\NormalTok{(}\SpecialCharTok{{-}}\FunctionTok{c}\NormalTok{(}
\NormalTok{    month, day, carrier, dest\_name, hour,}
\NormalTok{    minute, carrier\_name, arr\_delay}
\NormalTok{  ))}
\end{Highlighting}
\end{Shaded}

\begin{verbatim}
  dep_time dep_delay arr_time tailnum flight dest air_time distance
1     1403      1553     1934  N595AA   1568  DFW      182     1616
\end{verbatim}

We see that the flight occurred on March 3rd and departed a little after 2 PM on its way to Dallas/Fort Worth. Lastly, we show how we can visualize the arrival delay of all departing flights from Portland on March 3rd against time of departure.

\begin{Shaded}
\begin{Highlighting}[]
\NormalTok{flights }\SpecialCharTok{\%\textgreater{}\%}
  \FunctionTok{filter}\NormalTok{(month }\SpecialCharTok{==} \DecValTok{3}\NormalTok{, day }\SpecialCharTok{==} \DecValTok{3}\NormalTok{) }\SpecialCharTok{\%\textgreater{}\%}
  \FunctionTok{ggplot}\NormalTok{(}\FunctionTok{aes}\NormalTok{(}\AttributeTok{x =}\NormalTok{ dep\_time, }\AttributeTok{y =}\NormalTok{ arr\_delay)) }\SpecialCharTok{+}
  \FunctionTok{geom\_point}\NormalTok{()}
\end{Highlighting}
\end{Shaded}

\includegraphics{thesis_files/figure-latex/march3plot-1.pdf}

\section{Additional resources}\label{additional-resources}

\begin{itemize}
\item
  \emph{Markdown} Cheatsheet - \url{https://github.com/adam-p/markdown-here/wiki/Markdown-Cheatsheet}
\item
  \emph{R Markdown}

  \begin{itemize}
  \tightlist
  \item
    Reference Guide - \url{https://www.rstudio.com/wp-content/uploads/2015/03/rmarkdown-reference.pdf}
  \item
    Cheatsheet - \url{https://github.com/rstudio/cheatsheets/raw/master/rmarkdown-2.0.pdf}
  \end{itemize}
\item
  \emph{RStudio IDE}

  \begin{itemize}
  \tightlist
  \item
    Cheatsheet - \url{https://github.com/rstudio/cheatsheets/raw/master/rstudio-ide.pdf}
  \item
    Official website - \url{https://rstudio.com/products/rstudio/}
  \end{itemize}
\item
  Introduction to \texttt{dplyr} - \url{https://cran.rstudio.com/web/packages/dplyr/vignettes/dplyr.html}
\item
  \texttt{ggplot2}

  \begin{itemize}
  \tightlist
  \item
    Documentation - \url{https://ggplot2.tidyverse.org/}
  \item
    Cheatsheet - \url{https://github.com/rstudio/cheatsheets/raw/master/data-visualization-2.1.pdf}
  \end{itemize}
\end{itemize}

\section{Mathematics and Science}\label{math-sci}

\section{Math}\label{math}

\TeX~is the best way to typeset mathematics. Donald Knuth designed \TeX~when he got frustrated at how long it was taking the typesetters to finish his book, which contained a lot of mathematics. One nice feature of \emph{R Markdown} is its ability to read LaTeX code directly.

If you are doing a thesis that will involve lots of math, you will want to read the following section which has been commented out. If you're not going to use math, skip over or delete this next commented section.

\section{Chemistry 101: Symbols}\label{chemistry-101-symbols}

Chemical formulas will look best if they are not italicized. Get around math mode's automatic italicizing in LaTeX by using the argument \texttt{\$\textbackslash{}mathrm\{formula\ here\}\$}, with your formula inside the curly brackets. (Notice the use of the backticks here which enclose text that acts as code.)

So, \(\mathrm{Fe_2^{2+}Cr_2O_4}\) is written \texttt{\$\textbackslash{}mathrm\{Fe\_2\^{}\{2+\}Cr\_2O\_4\}\$}.

\noindent Exponent or Superscript: \(\mathrm{O^-}\)

\noindent Subscript: \(\mathrm{CH_4}\)

To stack numbers or letters as in \(\mathrm{Fe_2^{2+}}\), the subscript is defined first, and then the superscript is defined.

\noindent Bullet: CuCl \(\bullet\) \(\mathrm{7H_{2}O}\)

\noindent Delta: \(\Delta\)

\noindent Reaction Arrows: \(\longrightarrow\) or \(\xrightarrow{solution}\)

\noindent Resonance Arrows: \(\leftrightarrow\)

\noindent Reversible Reaction Arrows: \(\rightleftharpoons\)

\subsection{Typesetting reactions}\label{typesetting-reactions}

You may wish to put your reaction in an equation environment, which means that LaTeX will place the reaction where it fits and will number the equations for you.

\begin{equation}
  \mathrm{C_6H_{12}O_6  + 6O_2} \longrightarrow \mathrm{6CO_2 + 6H_2O}
  \label{eq:reaction}
\end{equation}

We can reference this combustion of glucose reaction via Equation \eqref{eq:reaction}.

\subsection{Other examples of reactions}\label{other-examples-of-reactions}

\(\mathrm{NH_4Cl_{(s)}}\) \(\rightleftharpoons\) \(\mathrm{NH_{3(g)}+HCl_{(g)}}\)

\noindent \(\mathrm{MeCH_2Br + Mg}\) \(\xrightarrow[below]{above}\) \(\mathrm{MeCH_2\bullet Mg \bullet Br}\)

\section{Physics}\label{physics}

Many of the symbols you will need can be found on the math page \url{https://web.reed.edu/cis/help/latex/math.html} and the Comprehensive LaTeX Symbol Guide (\url{https://mirror.utexas.edu/ctan/info/symbols/comprehensive/symbols-letter.pdf}).

\section{Biology}\label{biology}

You will probably find the resources at \url{https://www.lecb.ncifcrf.gov/~toms/latex.html} helpful, particularly the links to bsts for various journals. You may also be interested in TeXShade for nucleotide typesetting (\url{https://homepages.uni-tuebingen.de/beitz/txe.html}). Be sure to read the proceeding chapter on graphics and tables.

\chapter{Graphics, References, and Labels}\label{ref-labels}

\section{Figures}\label{figures}

If your thesis has a lot of figures, \emph{R Markdown} might behave better for you than that other word processor. One perk is that it will automatically number the figures accordingly in each chapter. You'll also be able to create a label for each figure, add a caption, and then reference the figure in a way similar to what we saw with tables earlier. If you label your figures, you can move the figures around and \emph{R Markdown} will automatically adjust the numbering for you. No need for you to remember! So that you don't have to get too far into LaTeX to do this, a couple \textbf{R} functions have been created for you to assist. You'll see their use below.

In the \textbf{R} chunk below, we will load in a picture stored as \texttt{reed.jpg} in our main directory. We then give it the caption of ``Reed logo'', the label of ``reedlogo'', and specify that this is a figure. Make note of the different \textbf{R} chunk options that are given in the R Markdown file (not shown in the knitted document).

Here is a reference to the Reed logo: Figure \ref{fig:reedlogo}. Note the use of the \texttt{fig:} code here. By naming the \textbf{R} chunk that contains the figure, we can then reference that figure later as done in the first sentence here. We can also specify the caption for the figure via the R chunk option \texttt{fig.cap}.

\clearpage

Below we will investigate how to save the output of an \textbf{R} plot and label it in a way similar to that done above. Recall the \texttt{flights} dataset from Chapter \ref{rmd-basics}. (Note that we've shown a different way to reference a section or chapter here.) We will next explore a bar graph with the mean flight departure delays by airline from Portland for 2014.

\clearpage

Next, we will explore the use of the \texttt{out.extra} chunk option, which can be used to shrink or expand an image loaded from a file by specifying \texttt{"scale=\ "}. Here we use the mathematical graph stored in the ``subdivision.pdf'' file.

\textbf{More Figure Stuff}

Lastly, we will explore how to rotate and enlarge figures using the \texttt{out.extra} chunk option. (Currently this only works in the PDF version of the book.)

\section{Footnotes and Endnotes}\label{footnotes-and-endnotes}

You might want to footnote something. \footnote{footnote text} The footnote will be in a smaller font and placed appropriately. Endnotes work in much the same way. More information can be found about both on the CUS site or feel free to reach out to \href{mailto:data@reed.edu}{\nolinkurl{data@reed.edu}}.

\section{Bibliographies}\label{bibliographies}

Of course you will need to cite things, and you will probably accumulate an armful of sources. There are a variety of tools available for creating a bibliography database (stored with the .bib extension). In addition to BibTeX suggested below, you may want to consider using the free and easy-to-use tool called Zotero. The Reed librarians have created Zotero documentation at \url{https://libguides.reed.edu/citation/zotero}. In addition, a tutorial is available from Middlebury College at \url{https://sites.middlebury.edu/zoteromiddlebury/}.

\emph{R Markdown} uses \emph{pandoc} (\url{https://pandoc.org/}) to build its bibliographies. One nice caveat of this is that you won't have to do a second compile to load in references as standard LaTeX requires. To cite references in your thesis (after creating your bibliography database), place the reference name inside square brackets and precede it by the ``at'' symbol. For example, here's a reference to a book about worrying: {[}@Molina1994{]}. This \texttt{Molina1994} entry appears in a file called \texttt{thesis.bib} in the \texttt{bib} folder. This bibliography database file was created by a program called BibTeX. You can call this file something else if you like (look at the YAML header in the main .Rmd file) and, by default, is to placed in the \texttt{bib} folder.

For more information about BibTeX and bibliographies, see our CUS site (\url{https://web.reed.edu/cis/help/latex/index.html})\footnote{@reedweb2007}. There are three pages on this topic: \emph{bibtex} (which talks about using BibTeX, at \url{https://web.reed.edu/cis/help/latex/bibtex.html}), \emph{bibtexstyles} (about how to find and use the bibliography style that best suits your needs, at \url{https://web.reed.edu/cis/help/latex/bibtexstyles.html}) and \emph{bibman} (which covers how to make and maintain a bibliography by hand, without BibTeX, at \url{https://web.reed.edu/cis/help/latex/bibman.html}). The last page will not be useful unless you have only a few sources.

If you look at the YAML header at the top of the main .Rmd file you can see that we can specify the style of the bibliography by referencing the appropriate csl file. You can download a variety of different style files at \url{https://www.zotero.org/styles}. Make sure to download the file into the csl folder.

\vfill

\textbf{Tips for Bibliographies}

\begin{itemize}
\tightlist
\item
  Like with thesis formatting, the sooner you start compiling your bibliography for something as large as thesis, the better. Typing in source after source is mind-numbing enough; do you really want to do it for hours on end in late April? Think of it as procrastination.
\item
  The cite key (a citation's label) needs to be unique from the other entries.
\item
  When you have more than one author or editor, you need to separate each author's name by the word ``and'' e.g.~\texttt{Author\ =\ \{Noble,\ Sam\ and\ Youngberg,\ Jessica\},}.
\item
  Bibliographies made using BibTeX (whether manually or using a manager) accept LaTeX markup, so you can italicize and add symbols as necessary.
\item
  To force capitalization in an article title or where all lowercase is generally used, bracket the capital letter in curly braces.
\item
  You can add a Reed Thesis citation\footnote{@noble2002} option. The best way to do this is to use the phdthesis type of citation, and use the optional ``type'' field to enter ``Reed thesis'' or ``Undergraduate thesis.''
\end{itemize}

\section{Anything else?}\label{anything-else}

If you'd like to see examples of other things in this template, please contact the Data @ Reed team (email \href{mailto:data@reed.edu}{\nolinkurl{data@reed.edu}}) with your suggestions. We love to see people using \emph{R Markdown} for their theses, and are happy to help.

\chapter{Paper 3}\label{paper3}

If we don't want Conclusion to have a chapter number next to it, we can add the \texttt{\{-\}} attribute.

\textbf{More info}

And here's some other random info: the first paragraph after a chapter title or section head \emph{shouldn't be} indented, because indents are to tell the reader that you're starting a new paragraph. Since that's obvious after a chapter or section title, proper typesetting doesn't add an indent there.

\chapter{Synthesis}\label{synthesis}

This thesis provides novel insights into the formation, composition, and stability of vivianite in intertidal
sediments. This chapter will begin with a summary of the key findings and their potential implications for the
role of vivianite in phosphorus (P) cycling in coastal sediments. The discussion will then extend to how the
presented findings may apply to various environmental systems and be relevant to other research fields. An
examination will follow, discussing how the developed methodology for in-situ studies in this thesis widens
our geochemical toolbox and opens various avenues to study in-situ Fe mineral transformation processes.
Additionally, this chapter suggests possible directions for future research.
\#\# Summary and discussion
Vivianite formation mechanisms

Methodological significance
\#\# Conclusions and outlook
Future research directions
Relevance to other research fields

\appendix

\chapter{The First Appendix}\label{the-first-appendix}

\backmatter

\chapter*{References}\label{references}
\addcontentsline{toc}{chapter}{References}

\markboth{References}{References}

\noindent

\phantomsection\label{refs}
\begin{CSLReferences}{0}{1}
\end{CSLReferences}

\setlength{\parindent}{-0.20in}

@heinrich2020

  \listoftables

  \listoffigures


% Index?

\end{document}
